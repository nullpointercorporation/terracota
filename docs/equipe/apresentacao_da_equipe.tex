\documentclass[11pt]{article}

\usepackage[utf8]{inputenc}
\usepackage[brazil]{babel}
\usepackage[section]{placeins}
\usepackage{graphicx}
\usepackage{hyperref}

\makeatletter
\renewcommand\@biblabel[1]{\textbf{#1.}} 
\renewcommand{\@listI}{\itemsep=0pt}

\begin{document}
\title{TerraCota \\
Documento de Apresentação da Equipe}
\author{NullPointer Corporation}
\date{}
\maketitle


\section{Integrantes}

\begin{enumerate}

\item André Coêlho

\begin{itemize}
\item \textbf{Função:} Artista
\item \textbf{Experiência prévia com desenvolvimento de jogos:}
	Experiências amadoras com Multimedia Fusion 2 e Game Maker. Experiência 
	de um nível maior com ROM Hacking (modificação de jogos já existentes), 
	em especial Super Mario World: frequentava uma comunidade na qual eu era 
	reverenciado como artista de gráficos (por algum motivo).
\item \textbf{Contato:} alcbcoelho@gmail.com
\end{itemize}

\item Pedro Braga

\begin{itemize}
\item \textbf{Função:} Artista
\item \textbf{Experiência prévia com desenvolvimento de jogos:}
	Alguns projetos pequenos de RPG Maker. Desenvolvi a ideia de um jogo
	de sobrevivência e estratégia para um projeto da Universidade durante
	o meu segundo semestre.
\item \textbf{Contato:} pedrobragav@gmail.com
\end{itemize}

\item Wendy Abreu

\begin{itemize}
\item \textbf{Função:} Artista
\item \textbf{Experiência prévia com desenvolvimento de jogos:}
	Projeto inacabado de um game rpg com animação e programação em flash. 
	Projeto pequeno de um jogo de naves animado e programado em flash.
\item \textbf{Contato:} jehnnywest@yahoo.com.br
\end{itemize}

\item José de Abreu

\begin{itemize}
\item \textbf{Função:} Músico
\item \textbf{Experiência prévia com desenvolvimento de jogos:}
	Nenhuma experiência.
\item \textbf{Contato:} abreubacelar@gmail.com
\end{itemize}

\item Álvaro Fernando

\begin{itemize}
\item \textbf{Função:} Desenvolvedor - Gerente
\item \textbf{Experiência prévia com desenvolvimento de jogos:} 	
	Projeto inacabado de um game Arcade para Android utilizando a engine de
	um amigo (LEngine - Luciano Prestes). Utilizei em 2010 um framework também
	para Android, chamado Corona SDK.
\item \textbf{Contato:} alvarofernandoms@gmail.com
\end{itemize}

\item Carlos Oliveira

\begin{itemize}
\item \textbf{Função:} Desenvolvedor - Game Designer
\item \textbf{Experiência prévia com desenvolvimento de jogos:}
	Breve desenvolvimento com RPG Maker, Game Maker e C++ com SDL.
\item \textbf{Contato:} carlospecter@gmail.com
\end{itemize}

\item Macartur de Sousa Carvalho

\begin{itemize}
\item \textbf{Função:} Desenvolvedor - Programador
\item \textbf{Experiência prévia com desenvolvimento de jogos:}
	Nenhuma experiência.
\item \textbf{Contato:} macartur.sc@gmail.com
\end{itemize}

\end{enumerate}

\section{Histórico da Equipe}
 Equipe formada no primeiro semestre de 2015. Os desenvolvedores são amigos
 de faculdade e fora dela, em projetos no laboratório LAPPIS. A se uniu com
 outros colegas do campus Darcy Ribeiro, que são amigos entre si e formam a
 equipe de artistas.
 
\section{Casos de Sucesso}
 Nenhum projeto lançado até o momento.
 
\section{Localização}
 Brasília - Distrito Federal.
 
\section{Nome Fantasia}
 \textbf{NullPointer Corporation} - Baseado em uma exceção lançada por algumas
 linguagens de programação quando um objeto é chamada mas não foi inicializado.
 Uma brincadeira que gerou muitas ideias para imagens da nossa logo, por exemplo,
 e gerou também algumas risadas.

\end{document}