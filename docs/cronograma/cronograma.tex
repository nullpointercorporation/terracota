%%%%%%%%%%%%%%%%%%%%%%%%%%%%%%%%%%%%%%%%%
% Thin Sectioned Essay
% LaTeX Template
% Version 1.0 (3/8/13)
%
% This template has been downloaded from:
% http://www.LaTeXTemplates.com
%
% Original Author:
% Nicolas Diaz (nsdiaz@uc.cl) with extensive modifications by:
% Vel (vel@latextemplates.com)
%
% License:
% CC BY-NC-SA 3.0 (http://creativecommons.org/licenses/by-nc-sa/3.0/)
%
%%%%%%%%%%%%%%%%%%%%%%%%%%%%%%%%%%%%%%%%%

%----------------------------------------------------------------------------------------
%	PACKAGES AND OTHER DOCUMENT CONFIGURATIONS
%----------------------------------------------------------------------------------------

\documentclass[a4paper, 11pt]{article} % Font size (can be 10pt, 11pt or 12pt) and paper size (remove a4paper for US letter paper)

\usepackage[utf8]{inputenc} % Set utf8 code
\usepackage[protrusion=true,expansion=true]{microtype} % Better typography
\usepackage{graphicx} % Required for including pictures
\usepackage{wrapfig} % Allows in-line images
\usepackage{geometry}
\usepackage{tabularx}
\usepackage[table]{xcolor}
\usepackage{booktabs}

\geometry{verbose,tmargin=1.5cm,bmargin=1.5cm,lmargin=1.5cm,rmargin=1.5cm}
\usepackage{pgfgantt}

\usepackage{mathpazo} % Use the Palatino font
\usepackage[T1]{fontenc} % Required for accented characters
\linespread{1.05} % Change line spacing here, Palatino benefits from a slight increase by default



\makeatletter
\renewcommand\@biblabel[1]{\textbf{#1.}} % Change the square brackets for each bibliography item from '[1]' to '1.'
\renewcommand{\@listI}{\itemsep=0pt} % Reduce the space between items in the itemize and enumerate environments and the bibliography

\renewcommand{\maketitle}{ % Customize the title - do not edit title and author name here, see the TITLE block below
\begin{center} % Center align
{\LARGE\@title} % Increase the font size of the title

\vspace{20pt} % Some vertical space between the title and author name

\end{center}
}

%----------------------------------------------------------------------------------------
%	TITLE
%----------------------------------------------------------------------------------------

\title{\textbf{Cronograma \\ TerraCota}} % Title

%----------------------------------------------------------------------------------------

\begin{document}

\maketitle % Print the title section

%----------------------------------------------------------------------------------------
%	DOC BODY
%----------------------------------------------------------------------------------------

\section*{Tabela de atividades}

\begin{table}[h]
\centering
\begin{tabular}{lccrc}
\toprule
\textbf{Atividade} & \textbf{Início} & \textbf{Conclusão} & \textbf{Progresso} & \textbf{Responsável} \\
\midrule
Formar da Equipe                     & 17/03 & 25/03 & 100\% & Equipe \\ %ok grantt
\rowcolor[gray]{0.9}
Versão Inicial do documento de visão & 17/03 & 04/04 & 100\% & Designer \\ %ok grantt
Brainstorm                           & 18/03 & 20/03 & 100\% & Gerente \\ %ok grantt
\rowcolor[gray]{0.9}
Documento de apresentação da equipe  & 25/03 & 10/04 & 100\% & Gerente \\ %ok grantt
Conceito do Jogo                     & 25/03 & 25/03 & 100\% & Gerente \\ %ok grantt
\rowcolor[gray]{0.9}
Ambiente de Desenvolvimento          & 29/03 & 31/03 & 100\% & Programador \\ %ok grantt
Laço principal                       & 29/03 & 09/04 & 100\% & Programador \\ %ok grantt
\rowcolor[gray]{0.9}
Janela                               & 07/04 & 09/04 & 100\% & Programador \\ %ok grantt
Cronograma                           & 08/04 & xx/xx & 50\%  & Gerente \\ %ok grantt
\rowcolor[gray]{0.9}
Plano de negócio                     & 08/04 & 22/04 & 100\% & Gerente \\ %ok grantt
Versão Final do Documento de Visão   & 13/04 & xx/xx & 0\%   & Designer \\ %ok grantt
\rowcolor[gray]{0.9}
Versão Inicial do GDD                & 13/04 & xx/xx & 0\%   & Designer \\ %ok grantt
Textos                               & 18/04 & xx/xx & 0\%   & Programador \\ %ok grantt
\rowcolor[gray]{0.9}
Módulo de Vídeo                      & 18/04 & xx/xx & 0\%   & Programador \\ %ok grantt
Low-Fi Sketches                      & 28/04 & 02/05 & 100\% & Gerente \\ %ok grantt
\rowcolor[gray]{0.9}
Animações                            & 02/05 & xx/xx & 0\%   & Programador \\ %ok grantt
Imagens                              & 06/05 & xx/xx & 0\%   & Programador \\ %ok grantt
\rowcolor[gray]{0.9}
Tela de Apresentação                 & 09/05 & 12/05 & 100\% & Programador \\ %ok grantt
GDD: Requisitos Tecnológicos         & xx/xx & xx/xx & 0\%   & Designer \\
\rowcolor[gray]{0.9}
GDD: Front End                       & xx/xx & xx/xx & 0\%   & Designer \\
GDD: Telas                           & xx/xx & xx/xx & 0\%   & Designer \\
\rowcolor[gray]{0.9}
Câmera e HUD                         & xx/xx & xx/xx & 0\%   & Designer \\
Personagem Principal                 & xx/xx & xx/xx & 0\%   & Designer \\
\rowcolor[gray]{0.9}
Hi-Fi Sketches                       & xx/xx & xx/xx & 0\%   & Gerente \\
Controles                            & xx/xx & xx/xx & 0\%   & Programador \\
\rowcolor[gray]{0.9}
Gameplay                             & xx/xx & xx/xx & 0\%   & Programador \\
Power Ups                            & xx/xx & xx/xx & 0\%   & Designer \\
\rowcolor[gray]{0.9}
Saúde                                & xx/xx & xx/xx & 0\%   & Designer \\
Sistema de Pontuação                 & xx/xx & xx/xx & 0\%   & Designer \\
\rowcolor[gray]{0.9}
Economia                             & xx/xx & xx/xx & 0\%   & Designer \\
Principais Personagens do Mundo do Jogo & xx/xx & xx/xx & 0\% & Designer \\
\rowcolor[gray]{0.9}
Veículos                             & xx/xx & xx/xx & 0\%   & Designer \\
\midrule
\end{tabular}
\end{table}

\newpage
\begin{table}[h]
\centering
\begin{tabular}{lccrc}
\toprule
\textbf{Atividade} & \textbf{Início} & \textbf{Conclusão} & \textbf{Progresso} & \textbf{Responsável} \\
\midrule
Progresso do Jogo                    & xx/xx & xx/xx & 0\%   & Designer \\
\rowcolor[gray]{0.9}
Visão Geral do Mundo do Jogo         & xx/xx & xx/xx & 0\%   & Designer \\
Mecânicas Universais                 & xx/xx & xx/xx & 0\%   & Designer \\
\rowcolor[gray]{0.9}
Níveis                               & xx/xx & xx/xx & 0\%   & Designer \\
Inimigos                             & xx/xx & xx/xx & 0\%   & Designer \\
\rowcolor[gray]{0.9}
NPCs                                 & xx/xx & xx/xx & 0\%   & Designer \\
Objetos Colecionáveis                & xx/xx & xx/xx & 0\%   & Designer \\
\rowcolor[gray]{0.9}
Minigames                            & xx/xx & xx/xx & 0\%   & Designer \\
Músicas e Efeitos Sonoros            & xx/xx & xx/xx & 0\%   & Designer \\
\rowcolor[gray]{0.9}
Músicas                              & xx/xx & xx/xx & 0\%   & Gerente \\
Efeitos Sonoros                      & xx/xx & xx/xx & 0\%   & Gerente \\
\rowcolor[gray]{0.9}
Arte Definitiva                      & xx/xx & xx/xx & 0\%   & Gerente \\
Instalador                           & xx/xx & xx/xx & 0\%   & Gerente \\
\rowcolor[gray]{0.9}
Manual                               & xx/xx & xx/xx & 0\%   & Gerente \\
Áudio                                & xx/xx & xx/xx & 0\%   & Programador \\
\rowcolor[gray]{0.9}
Rede                                 & xx/xx & xx/xx & 0\%   & Programador \\
Primeira Fase Jogável                & xx/xx & xx/xx & 0\%   & Programador \\
\midrule
\rowcolor[gray]{0.7}
\textbf{(Marco) MILESTONE 1 (30\% do jogo)} & 29/05 & 29/05 & 0\% & Equipe\\
\midrule
Versão Alfa                          & xx/xx & xx/xx & 0\%   & Programador \\
\midrule
\rowcolor[gray]{0.7}
\textbf{(Marco) MILESTONE 2 (70\% do jogo)} & 19/06 & 19/06 & 0\% & Equipe\\
\midrule
Versão Beta                          & xx/xx & xx/xx & 0\%   & Programador \\
\midrule
\rowcolor[gray]{0.7}
\textbf{(Marco) MILESTONE 3 (100\% do jogo)} & 10/07 & 10/07 & 0\% & Equipe\\
\midrule
                                     & xx/xx & xx/xx & 0\%   &      \\
\midrule
\end{tabular}
\end{table}

\newpage
\section*{Gráfico Gantt - Divido por semanas}
\definecolor{linkred}{RGB}{165,0,33}
\setganttlinklabel{f-s}{}
\begin{ganttchart}[
    canvas/.append style={fill=none, draw=black!15, line width=.75pt},
    hgrid style/.style={draw=black!5, line width=.75pt},
    vgrid={*1{draw=black!15, line width=.75pt}},
    today=11,
    today rule/.style={
      draw=black!64,
      dash pattern=on 3.5pt off 4.5pt,
      line width=1.5pt
    },
    today label font=\small\bfseries,
    title/.style={draw=none, fill=none},
    title label font=\bfseries\footnotesize,
    title label node/.append style={below=7pt},
    include title in canvas=false,
    bar label font=\mdseries\small\color{black!70},
    bar label node/.append style={left=.2cm},
    bar/.append style={draw=none, fill=black!63},
    bar incomplete/.append style={fill=barblue},
    bar progress label font=\mdseries\footnotesize\color{black!70},
    group incomplete/.append style={fill=groupblue},
    group left shift=0,
    group right shift=0,
    group height=.5,
    group peaks tip position=0,
    group label node/.append style={left=.6cm},
    group progress label font=\bfseries\small,
    link/.style={-latex, line width=1.5pt, linkred},
    link label font=\scriptsize\bfseries,
    link label node/.append style={below left=-2pt and 0pt}
  ]{1}{18}
 \gantttitle[
    title label node/.append style={below left=7pt and 2pt}
  ]{SEMANAS:\quad}{0}
  \gantttitlelist{1,...,18}{1} \\
  \ganttgroup{MILESTONE 1}{1}{12} \\
  \ganttbar[name=task1]{Brainstorm}{2}{2} \\
  \ganttbar[name=task2]{Formação da Equipe}{2}{3} \\
  \ganttbar[name=task3]{Versão Inicial do Documento de Visão}{2}{4} \\
  \ganttbar[name=task4]{Conceito do Jogo}{3}{3} \\
  \ganttbar[name=task5]{Ambiente de Desenvolvimento}{3}{3} \\
  \ganttbar[name=task6]{Laço Principal}{3}{5} \\
  \ganttbar[name=task7]{Documento de Apresentação da Equipe}{3}{9} \\
  \ganttbar[name=task8]{Janela}{5}{5} \\
  \ganttbar[name=task9]{Plano de Negócio}{5}{7} \\
  \ganttbar[name=task10]{Cronograma}{5}{11} \\
  \ganttbar[name=task11]{Versão Final do Documento de Visão}{6}{12} \\
  \ganttbar[name=task12]{Versão Inicial do GDD}{6}{12} \\
  \ganttbar[name=task13]{Textos}{6}{12} \\
  \ganttbar[name=task14]{Módulo de Vídeo}{6}{12} \\
  \ganttbar[name=task15]{Low-Fi Sketches}{7}{8} \\
  \ganttbar[name=task16]{Animações}{8}{12} \\
  \ganttbar[name=task17]{Tela de Apresentação}{9}{10} \\
  \ganttbar[name=task18]{Imagens}{9}{12} \\
  \ganttmilestone{Milestone 1 - Marco}{12} \\
  \ganttgroup{MILESTONE 2}{13}{15} \\
  \ganttgroup{MILESTONE 3}{16}{18} \\
  \ganttlink[link type=f-s]{task1}{task4}
  \ganttlink[link type=f-s]{task2}{task5}
  \ganttlink[link type=f-s]{task2}{task7}
  \ganttlink[link type=f-s]{task5}{task6}
  \ganttlink[link type=f-s]{task3}{task11}
  \ganttlink[link type=f-s]{task11}{task12}
  \ganttlink[link type=f-s]{task12}{task16}
  \ganttlink[link type=f-s]{task15}{task18}
\end{ganttchart}

\end{document}