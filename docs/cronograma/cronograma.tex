%%%%%%%%%%%%%%%%%%%%%%%%%%%%%%%%%%%%%%%%%
% Thin Sectioned Essay
% LaTeX Template
% Version 1.0 (3/8/13)
%
% This template has been downloaded from:
% http://www.LaTeXTemplates.com
%
% Original Author:
% Nicolas Diaz (nsdiaz@uc.cl) with extensive modifications by:
% Vel (vel@latextemplates.com)
%
% License:
% CC BY-NC-SA 3.0 (http://creativecommons.org/licenses/by-nc-sa/3.0/)
%
%%%%%%%%%%%%%%%%%%%%%%%%%%%%%%%%%%%%%%%%%

%----------------------------------------------------------------------------------------
%	PACKAGES AND OTHER DOCUMENT CONFIGURATIONS
%----------------------------------------------------------------------------------------

\documentclass[a4paper, 11pt]{article} % Font size (can be 10pt, 11pt or 12pt) and paper size (remove a4paper for US letter paper)

\usepackage[utf8]{inputenc} % Set utf8 code
\usepackage[protrusion=true,expansion=true]{microtype} % Better typography
\usepackage{graphicx} % Required for including pictures
\usepackage{wrapfig} % Allows in-line images
\usepackage[bottom=2cm,top=3cm,left=1.5cm,right=1.5cm]{geometry}

\usepackage{mathpazo} % Use the Palatino font
\usepackage[T1]{fontenc} % Required for accented characters
\linespread{1.05} % Change line spacing here, Palatino benefits from a slight increase by default

\makeatletter
\renewcommand\@biblabel[1]{\textbf{#1.}} % Change the square brackets for each bibliography item from '[1]' to '1.'
\renewcommand{\@listI}{\itemsep=0pt} % Reduce the space between items in the itemize and enumerate environments and the bibliography

\renewcommand{\maketitle}{ % Customize the title - do not edit title and author name here, see the TITLE block below
\begin{center} % Center align
{\LARGE\@title} % Increase the font size of the title

\vspace{20pt} % Some vertical space between the title and author name

\end{center}
}

%----------------------------------------------------------------------------------------
%	TITLE
%----------------------------------------------------------------------------------------

\title{\textbf{Cronograma \\ TerraCota}} % Title

%----------------------------------------------------------------------------------------

\begin{document}

\maketitle % Print the title section

%----------------------------------------------------------------------------------------
%	DOC BODY
%----------------------------------------------------------------------------------------

\section*{Tabela de atividades}

\begin{table}[h]
\begin{tabular}{|l|l|l|l|l|l|}
\hline
\textbf{Atividade}                           & \textbf{Início} & \textbf{Fim} & \textbf{Conclusão} & \textbf{Progresso} & \textbf{Responsável} \\ \hline
Formar da Equipe                             & 17/03           & 25/03        & 25/03              & 100\%              & Equipe              \\ \hline
Criar documento de visão                     & 17/03           & 04/04        & 04/04              & 100\%                & Designer             \\ \hline
Realizar Brainstorm                          & 18/03           & 20/03        & 20/03              & 100\%              & Gerente              \\ \hline
Criar documento de apresentação da equipe    & 25/03           & 08/04        & 		              & 70\%              & Gerente              \\ \hline
Definir Conceito do Jogo                     & 25/03           & 25/03        & 25/03              & 100\%               & Gerente              \\ \hline
Definir Ambiente de Desenvolvimento          & 29/03           & 31/03        & 31/03              & 100\%                & Programador          \\ \hline
Criar laço principal                         & 29/03           &              &                    & 80\%              & Programador          \\ \hline
Criar Janela                                 & 07/04           &              &                    & 0\%                & Programador          \\ \hline
Criar cronograma                             & 08/04           & 08/04        &   				 & 50\%              & Gerente              \\ \hline
Criar plano de negócio                       & 08/04           &              &                    & 0\%                & Gerente              \\ \hline
Criar GDD                                    & 10/04           & 10/05        &                    & 0\%                & Designer             \\ \hline
Módulo de Vídeo                     			&                 &              &                    & 0\%                & Programador          \\ \hline
\textbf{(Marco) MILESTONE 1 (30\% do jogo)}  & 29/05           & 29/05        &                    & 0\%                & Equipe		           \\ \hline
									        &          		  &         		&                    & 0\%                & 	     	           \\ \hline
\textbf{(Marco) MILESTONE 2 (70\% do jogo)}  & 19/06           & 19/06        &                    & 0\%                & Equipe		           \\ \hline
									        &          		  &         		&                    & 0\%                & 	     	           \\ \hline
\textbf{(Marco) MILESTONE 3 (100\% do jogo)} & 10/07           & 10/07        &                    & 0\%                & Equipe		           \\ \hline
									        &          		  &         		&                    & 0\%                & 	     	           \\ \hline
\end{tabular}
\end{table}

\end{document}