\documentclass[11pt]{article}

\usepackage[utf8]{inputenc}
\usepackage[brazil]{babel}
\usepackage[section]{placeins} 
\usepackage{graphicx}

\begin{document}
\title{TerraCota}
\author{NullPointer Corporation}
\date{}
\maketitle

\newpage

\tableofcontents
\newpage
\section{Apresentação}
TerraCota é um jogo do tipo aventura com uso de puzzle para progresso na história do jogo.  O protagonista irá se comunicar com os habitantes das torres através de símbolos, os quais necessitarão ser traduzidos para cumprir os objetivos das missões.

\section{Resumo do jogo}
Por conta de um desastre iminente, a humanidade construiu diversas torres auto-suficientes para se abrigar.
Centenas de anos depois, o mundo já re-equilibrado, a sociedade "evoluiu" a falar por meio de símbolos, ao invés de sons e palavras, e retornou às culturas de seus ancestrais.

Em uma das torres vive Inti, numa sociedade que esqueceu da existência do mundo de fora de sua torre.
Um dia, por algum motivo, ele acaba encontrando a saída da torre, e ao descobrir que existe algo além da torre, decide explorar.

Ele encontra Killa sentada nos galhos de uma árvore, e ela misteriosamente possui um interesse grande por ele.

\section{Principais Características}
Uma das características mais interessantes do jogo é o modo como os gêneros aventura e puzzle interagem entre si, fazendo com o que o jogador precise pensar e decifrar qual será sua missão, pois as informações serão dadas através de símbolos.

A movimentação do personagem, baseada em jogos \textit{Beat 'em Up}, também é uma característica diferenciada do jogo, pois trazemos um modo de exploração em visão isométrica. A inspiração desse estilo de movimentação veio, principalmente, do jogos Capitão Comando, Little Fighter 2 e Double Dragon.

\section{Público}
O jogo é voltado para o público geral por não possuir nenhum conteúdo que o faz ser censurado a nenhuma idade.
\section{Plataformas alvo}
As plataformas suportadas, inicialmente, serão Windows e Linux.
O Linux possui um grande suporte de bibliotecas e ferramentas de desenvolvimento de software e o Windows possui uma grande base de usuários jogadores, fazendo com que ambas os sistemas operacionais se tornem alvos para o jogo.

\section{Relação de logotipo}

\begin{figure}[!htp]
\centering
\includegraphics[scale=0.75]{logo-terracota.jpg}
\caption{Logo do jogo TerraCota}
\label{TerraCota logo}
\end{figure}

\section{Equipe}
\subsection{Contato por e-mail}
A equipe pode ser contatada através do email: \textit{nullpointercorporation@gmail.com}

\subsection{Contato por repositório}
Caso a intenção seja contatar a equipe para reportar algum \textit{bug} no jogo, o endereço a seguir deve ser acessado: https://github.com/nullpointercorporation/terracota/issues/
\newpage
\end{document}

